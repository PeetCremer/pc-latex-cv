\documentclass[singlesided,
               %doublesided,
               paper=a4,
               fontsize=10pt
              ]{my-resume}

%%%%%%%%%%%%%%%%%%%%%%%%%%%%%%%%%%%%%%%%%%%%%%%%%%%%%%%%%%%%%%%%%%%%%%%%%%%%%%%%
% set geometry
%%%%%%%%%%%%%%%%%%%%%%%%%%%%%%%%%%%%%%%%%%%%%%%%%%%%%%%%%%%%%%%%%%%%%%%%%%%%%%%%

%\setlength\headerheight{4cm}            % note that margintop gets added to this value, i.e. the header bar is 5cm
\setlength\marginleft{1cm}
\setlength\marginright{\marginleft}      % needs to be 1.5 times to be actually equal. why?
\setlength\margintop{1cm}
\setlength\marginbottom{1cm}


%%%%%%%%%%%%%%%%%%%%%%%%%%%%%%%%%%%%%%%%%%%%%%%%%%%%%%%%%%%%%%%%%%%%%%%%%%%%%%%%
% FONTS
%%%%%%%%%%%%%%%%%%%%%%%%%%%%%%%%%%%%%%%%%%%%%%%%%%%%%%%%%%%%%%%%%%%%%%%%%%%%%%%%

\RequirePackage{fontspec}
\setmainfont{Carlito}


%%%%%%%%%%%%%%%%%%%%%%%%%%%%%%%%%%%%%%%%%%%%%%%%%%%%%%%%%%%%%%%%%%%%%%%%%%%%%%%%
% COLORS
%%%%%%%%%%%%%%%%%%%%%%%%%%%%%%%%%%%%%%%%%%%%%%%%%%%%%%%%%%%%%%%%%%%%%%%%%%%%%%%%

\colorlet{highlightbarcolor}{lightgray}
\colorlet{headerbarcolor}{darkgray}

\colorlet{headerfontcolor}{white}
\colorlet{accent}{awesome-katzi3}
\colorlet{heading}{black}
\colorlet{emphasis}{black}
\colorlet{body}{black}

%%%%%%%%%%%%%%%%%%%%%%%%%%%%%%%%%%%%%%%%%%%%%%%%%%%%%%%%%%%%%%%%%%%%%%%%%%%%%%%%
% Additional Packages
%%%%%%%%%%%%%%%%%%%%%%%%%%%%%%%%%%%%%%%%%%%%%%%%%%%%%%%%%%%%%%%%%%%%%%%%%%%%%%%%


\usepackage{hyperref}
\hypersetup{colorlinks=true,linkcolor=awesome-skyblue,urlcolor=awesome-skyblue}

\usepackage{cleveref}[2012/02/15]% v0.18.4; 
\crefformat{footnote}{#2\footnotemark[#1]#3}



%%%%%%%%%%%%%%%%%%%%%%%%%%%%%%%%%%%%%%%%%%%%%%%%%%%%%%%%%%%%%%%%%%%%%%%%%%%%%%%%
% set document
%%%%%%%%%%%%%%%%%%%%%%%%%%%%%%%%%%%%%%%%%%%%%%%%%%%%%%%%%%%%%%%%%%%%%%%%%%%%%%%%

\begin{document}

\pagestyle{headermain}

\firstname{Dr. Peet}\lastname{Cremer}

\address{Transistorfaret 9, 1396 Billingstad, Norway}
\birthday{27/01/1988}
\phone{+47 917 42 339}
\email{peet.cremer@gmail.com}
\linkedin{Peet Cremer}{https://www.linkedin.com/in/peet-cremer-5286601a6/}
\github{@PeetCremer}{https://github.com/PeetCremer}
\googlescholar{Publication list}{https://scholar.google.com/citations?hl=de&user=MnU8ZxwAAAAJ}
%\homepage{peetcremer.com}{https://www.peetcremer.de}
%\orcid{0000-0001-6040-6596}{https://orcid.org/0000-0001-6040-6596}
%\ads{NASA/ADS publication list}{https://ui.adsabs.harvard.edu/search/fq=\%7B!type\%3Daqp\%20v\%3D\%24fq\_database\%7D&fq\_database=database\%3A\%20astronomy&p\_=0&q=pubdate\%3A\%5B2016-01\%20TO\%209999-12\%5D\%20author\%3A(\%22Krieger\%2C\%20Nico\%22)&sort=date\%20desc\%2C\%20bibcode\%20desc}


\tagline{Enthusiastic Engineering Manager with a can-do attitude and experience in leading teams of up to 15 people. My technical background is in Statistical Physics and Machine Learning. 
I combine advanced software engineering skills (Python, C++, Rust) with the knowledge of creating Machine Learning pipelines and of building up scalable cloud data platforms.
I have an outgoing personality, build trust through honesty, employ strategic thinking, and base decisions on facts and data.

TODO this needs some brushup with the recent change back to an IC role. Probably should emphasize the hands-on work but also the strategic thinking and general qualities as an engineering leader through the experience as Engineering Manager.}
\photo[round]{newpic3.jpg}{\dimexpr \headerheight-\marginbottom-1em}   % make photo exactly match the header with margintop/marginright/marginbottom as margin

\makeheader

\section[\faBriefcase]{Work experience}
\job{Principal Engineer -- Atlas AI}
	{COGNITE (Oslo, Norway)}
    {02/2023 - 2/2024}
    {
        TODO

        I recently changed positions within my company (still Cognite). 
        I am now "Principal Engineer - Atlas AI" since February 2025.
        Information what Atlas AI is about heree: https://www.cognite.com/en/product/atlas . 
        Within Atlas AI, my role is to lead the creation of evaluation frameworks to measure how well Cognite's AI systems actually perform. 
        That is measure Atlas AI industrial agents, but also the underlying natural language tools which they use to interface with Cognite's data platform.

        I am also doing pioneer work in identifying productivity enhancing AI tools for the organization, such as Cursor, Bolt Claude Code, Gemini Code Assist, and various MCP servers.
        I am taking holistic ownership here, not only testing the tools but also working with Legal and Security to ensure that we are compliant with the laws and regulations, and that potential security risks are mitigated and managed.
        For Cursor, I led the procurement process end-to-end and worked effectively through the organization to ensure compliance, remove blockers, and get it rolled out to the organization.
    }
%    
\job{Senior Director of Engineering}
	{COGNITE (Oslo, Norway)}
    {02/2024 - 2/2025}
    {
        TODO flesh out more. This is a more Senior Engineering Leadership role then below.
        Although the team setups are the same. Team sizes hovering around 10-15 developers
        However, additionally led a hiring push in India to grow Cognite's engineering organization there. 
        Also helped develop the product strategy for Contextualization at Cognite.
        Also participated in culture building
        Also did some strategic Senior Engineering Leadership to help Cognite's engineering organization grow. 
        \begin{itemize}[leftmargin=2em]
            \item Team sizes varying between 5 and 17 developers focusing on data integration, \href{https://www.cognite.com/en/contextualization}{data connectivity}, and \href{https://www.cognite.com/en/product/atlas}{AI}
            \item People management, product leadership and strategy, culture building, and technical architecture.
        \end{itemize}
    }
%
\job{Director of Engineering}
	{COGNITE (Oslo, Norway)}
	{02/2023 - today}
    {
        Engineering manager of 18 developers in 3 teams, focusing on:
        \begin{itemize}[leftmargin=2em]
            \item \href{https://www.cognite.com/en/contextualization}{contextualization of industrial data}
            \item \href{https://www.cognite.com/en/industrial-canvas}{data-driven troubleshooting apps}
            \item \href{https://docs.cognite.com/cdf/integration/guides/contextualization/interactive_diagrams/}{Parsing of engineering diagrams for industrial sites}
        \end{itemize}

        TODO: should emphasize somewhat that I started with 1 team of 5 developers and grew to around 15 within this time quite quickly.
        The 18 developers in 3 teams where only at the spike. Most times I had 2 teams. The number of reports was high for most of the time though.
        1 team always stayed constant, but the other 2 teams where fluctuating. I was pretty much filling the gaps where needed.
    }
%
\job{Senior Machine Learning Engineer and Tech Lead}
	{COGNITE (Oslo, Norway)}
	{08/2021 - 02/2023}
    {\begin{itemize}[leftmargin=2em]
        \item Leading a cross-functional team of 5 software / ML engineers
        \item Implementing intelligent algorithms to find context in otherwise unstructured industrial data
        \item Scaling and maintaining microservices to deploy those algorithms in an SaaS setting
        \item Creating data infrastructure capabilities to build up an industrial knowledge graph
    \end{itemize}}
%
\job{AI Lead Developer}
    {APTIV (Wuppertal, Germany)}
    {12/2020 - 07/2021}
    {\begin{itemize}[leftmargin=2em]
        \item Planning and execution of Machine Learning and Data Infrastructure projects in the automotive industry
        \item Design of AI solutions for automotive perception tasks. Guiding the software and hardware integration into the test vehicle
        \item Participated in a lot of innovation, leading to 7 patents and 1 publication (see \href{https://scholar.google.com/citations?hl=de&user=MnU8ZxwAAAAJ}{publication list})
    \end{itemize}}
%
\job{Software Development Expert}
    {APTIV (Wuppertal, Germany)}
    {07/2017 - 12/2020}
    {\begin{itemize}[leftmargin=2em]
        \item Leading development of a data platform for storage and retrieval of automotive sensor data as a product owner
        \item Development of infrastructure solutions for artifical intelligence in automotive applications 
        \item Established a microservice architecture to automate AI workflows
        \item Supervision of a Master Thesis on using GANs for automotive data style transfer
    \end{itemize}}


\section[\faMortarBoard]{Education}
\job{Doctor (Ph.D.), Theoretical Soft Matter Physics}
    {University of Düsseldorf}
    {2013 - 2017}
    {
        \begin{itemize}[leftmargin=2em]
            \item Topic: Mesoscale modeling of magnetic elastomers and gels -- theory and simulations
            \item Solving magneto-elastic coupling models using numerical simulations, the finite element method, and density functional theory
            \item Resulted in 7 publications in recognized peer-reviewed journals (see \href{https://scholar.google.com/citations?hl=de&user=MnU8ZxwAAAAJ}{publication list})
        \end{itemize}
    }
%
\job{Master of Science (M. Sc.), Physics}
    {University of Düsseldorf}
    {2012 - 2013}
    {
        \begin{itemize}[leftmargin=2em]
            \item \textbf{Gpa: 1.1} (grades at german university range from 1.0 (best) to 4.0 (worst)). Minor: Mathematics
            \item Focus on Soft Matter, Plasma Physics, Solid-State and Nanophysics
            \item Master thesis: "Emergent states in active systems" was published as a \href{https://journals.aps.org/pre/abstract/10.1103/PhysRevE.89.022307}{journal article}
        \end{itemize}
    }
%
\job{Bachelor of Science (B. Sc.), Physics}
    {University of Düsseldorf}
    {2008 - 2012}
    {
        \begin{itemize}[leftmargin=2em]
            \item \textbf{Gpa: 1.2} (grades at german university range from 1.0 (best) to 4.0 (worst)). Minor: Mathematics
            \item Bachelor thesis: "Orientational fields in Plastic Crystals" was published as a \href{https://epljournal.edpsciences.org/articles/epl/abs/2012/15/epl14756/epl14756.html}{journal article}.
        \end{itemize}
    }


\section[\faFlask]{Skills}
\skilllist{Programming languages}{\tag{Python} \tag{C++} \tag{C} \tag{Rust} \tag{Typescript}}
\skilllist{IT working knowledge}{\tag{Docker} \tag{GitLab} \tag{GitHub} \tag{VS Code} \tag{MongoDB} \tag{PostgreSQL} \tag{flask} \tag{FastAPI} \tag{Kubernetes} \tag{Grafana} \tag{REST} \tag{Linux} \tag{Node.js} \tag{Azure} \tag{GCP} \tag{\LaTeX} \tag{MS Office}}
\skilllist{Libraries and frameworks}{\tag{TensorFlow} \tag{PyTorch} \tag{scikit-learn} \tag{fastAPI} \tag{flask} \tag{numpy} \tag{scipy} \tag{Qt} \tag{pandas} \tag{OpenMP}}
%
\clearpage
\pagestyle{empty}
%
\skilllist{Machine Learning techniques}{\tag {SVMs} \tag{Gradient Boosting} \tag{Evolutionary algorithms} \tag{Decision Trees} \tag{CNNs}}
\skilllist{Agile software development}{\tag{Scrum} \tag{Kanban} \tag{Jira} \tag{Confluence}}
\skilllist{Languages}{\tag{German (native)} \tag{English (C1)} \tag{Norwegian (B2)} \tag{French (A2)}}
%\smallskip % additional skip because tag outlines use up space



\section[\faTrophy]{Achievements, honours, and awards}
\achievement{Best Poster presentation at the 15\textsuperscript{th} German Ferrofluid Workshop in Rostock (2015).}
\achievement{DAAD scholarship ''RISE in North America`` for a three month research internship at Yale University, CT (2010)}


\section[\faGears]{Projects I contributed to as Manager or Individual Contributor}


%
\job{Finetuning of Language Model on Natural Language Querying (NLQ)}
	{COGNITE, Oslo, Norway}
	{02/2025 - 04/2025}
    {
        TODO need to refine this, make this more concise while keeping the important detials relevant for a CV. Perhaps also improve title.

        Together with a colleague, I finetuned a language model to perform better at natural language querying (NLQ) tasks on Cognite's knowledge graph.
        This was done by creating a curated dataset of NLQ problems and their ground truth answers.

        Cognite's CEO Girish Rishi made a LinkedIn post about this: 
        \href{https://www.linkedin.com/posts/girish-rishi-4392587_atlas-ai-is-all-about-enabling-our-customers-activity-7336752688279207936-f58Y?utm_source=share&utm_medium=member_desktop&rcm=ACoAADAmkPABVhPkjuk0vvCtiRSid_p7fC4zU_o}{post}
    }
%
\job{Benchmark of various LLM models on Cognite Atlas AI features}
	{COGNITE, Oslo, Norway}
	{02/2025 - 04/2025}
    {
        TODO need to refine this, make this more concise while keeping the important detials relevant for a CV. Perhaps also improve title.

        I led an effort to benchmark on the performance of various LLM models on Cognite Atlas AI features, specifically on the ability to do natural language querying (NLQ) 
        on Cognite's knowledge graph as well as Document Question Answering (DQA).
        This work involved creating hundreds of curated test cases, setting up evaluation infrastructure, and running the evaluations on various SOTA models from Anthropic, OpenAi, Google, Deepseek, and Mistral.
        Then, analyzing and interpreting the results. Both instruction tuned and reasoning models were evaluated. for their performance in those agentic tools.

        This culminated in an external benchmark report on the performance of various LLM models on Cognite Atlas AI features, link here: \href{https://www.cognite.com/en/resources/white-papers/atlas-ai-slm-llm-benchmark-report}{link to report}
        This report was used to inform the decision on which LLM models to use to power the features of Cognite Atlas AI agents going forward.
    }
%
\job{AI tooling for Cognite's organization}
	{COGNITE, Oslo, Norway}
	{03/2025 - 06/2025}
    {
        TODO need to refine this, make this more concise while keeping the important detials relevant for a CV. Perhaps also improve title.

        Identified key productivity enhancing AI tools for the organization, that are must have for a proper AI transition. 
        Basically I tested the tools out first personally to judge their maturity and suitability for Cognite's engineering organization.
        
        I took holistic ownership here: 
        - I tested tools such as Cursor, Windsurf, Bolt, Claude Code, Gemini Code Assist, various Model Context Protorocol (MCP) servers (Atlassian, GitHub, context7).
        - I identified which are must have and started procuring this, leading with Cursor. 
        - I worked with Legal and Security to ensure that we are compliant with the laws and regulations, and that potential security risks are mitigated and managed.
        - I led the purchasing process working with the procurement team to ensure a smooth and efficient rollout. 
        - I worked with architects and security engineers to ensure that the necessary infrastructure and security guardrails are in place to support the tools.
        - I conducted trainings and information sessions for engineers to know how to use the tools effectively and securely.

        Also gave a talk on this in a Global Toww Hall, which was praised by the Cognite's CEO Girish Rishi in a LinkedIn post: 
        \href{https://www.linkedin.com/posts/girish-rishi-4392587_peet-cremer-our-engineering-leader-showing-activity-7343289548442955776-Zu2p?utm_source=share&utm_medium=member_desktop&rcm=ACoAADAmkPABVhPkjuk0vvCtiRSid_p7fC4zU_o}{post}
    }
%
\job{Hiring push for Cognite's India Center of Excellence}
	{COGNITE, Oslo, Norway}
	{06/2024 - 12/2024}
    {
        TODO need to refine this, make this more concise while keeping the important detials relevant for a CV. Should maybe also change title.
        Basically what I did here was the following: There was a plan to open a new engineering center in India.
        However, there was no concrete plan on how to hire for this, what teams to form there, what roles to fill ect.
        So I was part of a small strategic committee to define a staffing plan for this new center.
        I was then leading the hiring process for the engineering roles in that center. This involved created new job descriptions, 
        interview processes, interview tasks, take home assignments. I also distributed different hiring pipelines to other managers in the organization
        and held a regular sync to track progress and hiring goals. In total, this was for about 20 new roles in engineering.

        Here is a press release from Cognite about the inauguration of the India Center of Excellence.
        \href{https://www.cognite.com/en/company/newsroom/cognite-inaugurates-india-center-of-excellence-in-bengaluru-with-commitment-to-leveraging-ai-for-industrial-growth}{press release}
    }
%
\job{Vectorstore for retrieval augmented generation}
	{COGNITE, Oslo, Norway}
	{04/2023 - 08/2023}
    {\begin{itemize}[leftmargin=2em]
		\item Vector similarity lookup service build on top of the \href{https://weaviate.io/}{Weaviate} vector database
		\item Enables to retrieve relevant context for LLM queries to enable an industrial chatbot and code completion experience
	\end{itemize}}
%
\job{Data backend to store an industrial knowledge graph}
	{COGNITE, Oslo, Norway}
	{06/2022 - 08/2022}
    {\begin{itemize}[leftmargin=2em]
		\item Creating a backend to store symbols and process lines extracted from engineering diagrams in an industrial knowledge graph
		\item Implemented in Python and Typescript and interfaces to COGNITE`s internal flexible data modeling service
		\item Allows for advanced graph queries on the knowledge graph and, thereby, enables advanced interactions with the industrial reality
	\end{itemize}}
%
\job{Annotation API to store auxiliary label data on files}
	{COGNITE, Oslo, Norway}
	{08/2021 - 06/2022}
    {\begin{itemize}[leftmargin=2em]
		\item Implemented a REST API to store label information on files within COGNITE`s data warehouse
		\item Went from design to fully productive usage with SLAs in less than a year
		\item Implemented in Python on top of \textit{PostgreSQL} using \textit{SQLalchemy} and \textit{flask}. Flexible annotation type system enabled by \textit{pydantic}
	\end{itemize}}
%
\job{Intelligent document scanning tool}
	{COGNITE, Oslo, Norway}
	{03/2021 - 06/2022}
    {\begin{itemize}[leftmargin=2em]
		\item Contributed to a document scannnig tool that detects relevant fields in scanned forms and automatically extracts their values, significantly reducing the human effort required
		\item Using Azure OCR to detect text instances together with a line detection algorithm to extract tables and fields. Combined with hand-crafted rules to make the field extraction more robust
	\end{itemize}}	
%
\job{Live execution of detection network in test vehicle}
    {APTIV, Wuppertal, Germany}
    {11/2020 - 12/2020}
    {\begin{itemize}[leftmargin=2em]
        \item Deployed a 3d bounding box detection network on Nvidia Jetson Xavier hardware 
        \item Optimizations and tweaks to make an automotive detection network fast enough to run live in the test vehicle
    \end{itemize}}
%
\job{Runtime environment for AI algorithms}
    {APTIV, Wuppertal, Germany}
    {06/2020 - 10/2020}
    {\begin{itemize}[leftmargin=2em]
        \item Runtime environment written in Rust for live execution of AI algorithms in test vehicles for demo purposes
        \item Main contributions: Preprocessing from the raw sensor data into the TensorFlow network input, subsequent postprocessing of the network results into bounding boxes for visualization, as well as abstractions to allow for different combinations of sensors and networks
    \end{itemize}}
%
 \job{Tooling for neural network training}
    {APTIV, Wuppertal, Germany}
    {02/2020 - 03/2020}
    {\begin{itemize}[leftmargin=2em]
        \item Python / Rust tooling to download sensor data and ground truth from a data warehouse and refine it for neural network training
        \item Sophisticated interpolation algorithm for 3d bounding boxes to arbitrary timestamps
        \item Using HDF5 as final data exchange format
    \end{itemize}}
%
\job{Machine Learning automation using microservices}
    {APTIV, Wuppertal, Germany}
    {03/2020 - 05/2020}
    {\begin{itemize}[leftmargin=2em]
        \item Established a Python microservice framework for the automatic execution of Machine Learning algorithms
        \item Automatic triggering of execution pipelines on trigger events, such as the availability of new data
    \end{itemize}}
%
\job{Deploying a facial expression detection system}
    {Affectiva, Boston, MA}
    {08/2019}
    {\begin{itemize}[leftmargin=2em]
        \item Short-noticed support of cooperation partner Affectiva in Boston to mitigate risk in a customer project
        \item Made key contributions for deploying a facial expression detection system using TensorFlow and TF-Lite
    \end{itemize}}
%
\job{Product Owner for a data warehouse project}
    {APTIV, Wuppertal, Germany}
    {02/2019 - 02/2020}
    {\begin{itemize}[leftmargin=2em]
        \item Lead of a SCRUM team of 5 developers to establish a data warehouse for automotive sensor data and algorithm results
        \item Access to automotive driving scenarios for the development of AI-based driver assistance systems
        \item Based on MEAN stack, hosted in Azure using BlobStorage for larger binary data. Orchestrated using docker-compose
        \item Featuring a REST API, a Python access client, a frontend with a video playback tool, and full backend test coverage
    \end{itemize}}
%
\job{3D object detection on automotive radar data}
    {APTIV, Wuppertal, Germany}
    {12/2018 - 01/2019}
    {\begin{itemize}[leftmargin=2em]
        \item Lead a team of 5 engineers for a Deep Learning proof of concept
        \item Successfully demonstrated an anchor-based 3D object detection on automotive radar raw data using CNNs
    \end{itemize}}
%
\clearpage
\pagestyle{empty}
%
\job{Automotive recording tool}
    {APTIV, Wuppertal, Germany}
    {07/2018 - 08/2018}
    {\begin{itemize}[leftmargin=2em]
        \item Development of a tool using C++ and Qt for the recording of sensor data in a test vehicle.
        \item Recording of LiDAR (via UDP), Vehicle host bus and radar detections (via CAN), and radar debug information (via UDP)
        \item Emphasis on correct timestamping of recorded sensor data, such that it can be replayed after recording 
    \end{itemize}}
%
\job{LiDAR labeling tool}
    {APTIV, Wuppertal, Germany}
    {01/2018 - 12/2018}
    {\begin{itemize}[leftmargin=2em]
        \item Work on a web-based labeling tool for 3D bounding boxes in LiDAR point clouds using TypeScript
        \item Backend development using MEAN stack (MongoDB, Express, Angular, Node.js)
        \item Main contributions: User and group management and data upload
    \end{itemize}}
%
\job{Simple raytracer to simulate FMCW Radars}
    {APTIV, Wuppertal, Germany}
    {11/2017 - 12/2017}
    {\begin{itemize}[leftmargin=2em]
        \item Simulated an automotive FMCW radar by creating a simple raytracer in Python.
        \item Used this raytracer to simulate artificical training data for neural networks 
    \end{itemize}}
%
\job{Automatic code generator for CNNs}
    {APTIV, Wuppertal, Germany}
    {07/2017 - 10/2017}
    {\begin{itemize}[leftmargin=2em]
        \item Implemented code generator in Matlab to deploy CNNs to a TI embedded chip
        \item Given a CNN trained in TensorFlow, this generator creates optimized C++ code to execute that CNN on the target platform 
    \end{itemize}}

\section[\faBook]{Teaching}
\job{Co-Organizer of the NorwAI 2022 hackathon}
	{NTNU Trondheim,\\ Norway}
	{08/2022 - 10/2022}
    {\begin{itemize}[leftmargin=2em]
		\item Organizing and conducting a Data Science hackathon in Trondheim with Cognite and researchers from NTNU
		\item Finding a suitable dataset, defining a task, supervising the students during the event, and evaluating the contributions
	\end{itemize}}

\job{Lecturer on \\ artificial intelligence in autonomous driving}
    {University of Wuppertal,\\ Germany}
    {10/2020 - 04/2021}
    {\begin{itemize}[leftmargin=2em]
        \item Lecture "Artificial Intelligence Based Sensor Signal Processing for Autonomous Driving" held in collaboration with colleagues from APTIV
        \item Prepared and held lectures and exercises about Numerical Optimization in Data Science, Support Vector Machines, and Gradient Boosting
    \end{itemize}}
%
\job{Master thesis supervision}
    {APTIV, Wuppertal, Germany}
    {03/2019 - 09/2019}
    {\begin{itemize}[leftmargin=2em]
        \item Supervised a master student on using GANs for automotive data style transfer
        \item Created artificial LiDAR data by modding the video game GTA: V, then trained a GAN on real LiDAR data to do the domain transform
        \item Tested and benchmarked this approach with a birds-eye-view 2D object detection model
    \end{itemize}}
%
\job{Bachelor thesis supervision}
    {University of Düsseldorf, \\ Germany}
    {2016}
    {\begin{itemize}[leftmargin=2em]
        \item Supervised a bachelor student on the numerical simulation of magnetic gels 
    \end{itemize}}
%
\job{Teaching assistant for theoretical physics lectures}
    {University of Düsseldorf, \\ Germany}
    {2013-2017}
    {\begin{itemize}[leftmargin=2em]
        \item Lectures: Quantum Mechanics and Statistical Mechanics
        \item Created homeworks and gave exercise classes
        \item Answered student questions about the lecture topics
        \item Designed and held oral and written exams
    \end{itemize}}


\section[\faMusic]{About Me}
\hobbylist{Interests}{I am enthusiastic about AI, tech and science related topics. To follow the recent developments in machine learning, I like to read papers on arXiv and from the ICLR conference and I follow \href{https://towardsdatascience.com/}{towardsdatascience} and the \href{https://www.reddit.com/r/MachineLearning/}{/r/MachineLearning} subreddit. To stay on top of new trends in software engineering and science topics, I regularly browse \href{https://news.ycombinator.com/}{Hacker News}. Additionally I like to improve my leadership and organization skills by reading related books.}
\vspace{0.5em}
\hobbylist{Activities}{Sozializing with friends has always been important to me. I am an enthusiastic Pen \& Paper gamemaster since 20 years and often meet with friends to indulge together in this hobby. Keeping myself healthy with a good diet and regular exercise is another priority for me. To achieve this, I like to cook quality food with fresh ingredients, and I go running several times a week. To keep myself in shape and the environment clean, I take my racing bike to reach places whenever possible}


    
\end{document}